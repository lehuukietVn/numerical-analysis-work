\chapter{Tính gần đúng đạo hàm và tích phân}
\textbf{\color{blue}Bài 3.} Bằng phương pháp hình thang và Simpson 1/3 với $n=10$ để tính gần đúng và đánh giá sai số các tích phân sau:\par

b) $I=\int_{0}^{\pi} \sin x\mathrm{d}x$
Ta có $h=\frac{\pi-0}{10}=\frac{\pi}{10}$

\textbf{Áp dụng công thức Simpson 1/3}

Ta có bảng sau:
\begin{center}
\begin{tabular}{|c|c|c|c|c|}
\hline
$i$&$x_i$&\multicolumn {3} {|c|} {$y_i=f(x_i)= \sin x$}\\ \hline
0&0&0& & \\ \hline
1&$\frac{\pi}{10}$& &0,3090& \\ \hline
2&$\frac{\pi}{5}$& & &0,5878 \\ \hline
3&$\frac{3\pi}{10}$& &0,8090& \\ \hline
4&$\frac{2\pi}{5}$& & &0,9511 \\ \hline
5&$\frac{\pi}{2}$& &1& \\ \hline
6&$\frac{3\pi}{5}$& & &0,9511 \\ \hline
7&$\frac{7\pi}{10}$& &0,8090& \\ \hline
8&$\frac{4\pi}{5}$& & &0,5878 \\ \hline
9&$\frac{9\pi}{10}$& & 0,3090& \\ \hline
10&$\pi$&0& & \\ \hline
\end{tabular}
\end{center}

Theo công thức Simpson 1/3, ta có:

$I_S\approx \frac{\pi}{30}.\left[0+4.3,2361+2.3,0777\right]\approx 2,000105435$

Đánh giá sai số:

$$\lvert I-I_S \rvert\leqslant \frac{\max_{0\leqslant x\leqslant\pi}\left\lvert f^{(4)}(x)\right\rvert}{180}(\pi -0)h^4=\frac{1}{180}.\pi.\left( \frac{\pi}{10}\right)^4\approx 0,00017 $$

\textbf{Áp dụng công thức hình thang}

Ta được bảng sau:
\begin{center}
\begin{tabular}{|c|c|c|}
\hline
$x_i$&\multicolumn {2} {|c|} {$y_i=f(x_i)= \sin x$}\\ \hline
0&	0&	\\ \hline
$\frac{\pi}{10}$&		&0,3090\\ \hline
$\frac{\pi}{5}$&		&0,5878\\ \hline
$\frac{3\pi}{10}$&		&0,8090\\ \hline
$\frac{2\pi}{5}$&		&0,9511\\ \hline
$\frac{\pi}{2}$&		&1\\ \hline
$\frac{3\pi}{5}$&		&0,9511\\ \hline
$\frac{7\pi}{10}$&		&0,8090\\ \hline
$\frac{4\pi}{5}$&		&0,5878\\ \hline
$\frac{9\pi}{10}$&		&0,3090\\ \hline
1&	0&	\\ \hline
&0&6,3138 \\ \hline
\end{tabular}
\end{center}

Do đó giá trị gần đúng của tích phân đã cho là:
$$I_T\approx \frac{\pi}{2.10}.(0+2.6,3138)\approx 1,9835$$

*Đánh giá sai số:

Ta có $M=\max_{0\leqslant x\leqslant\pi}\lvert f''(x)\rvert=1$ và $\bar{I}=1,98$

nên $\lvert I_T -\bar{I}\rvert\leqslant \frac{M}{12}.(\pi -0).\left(\frac{\pi}{10}\right)^2\approx 0,026$

và $\lvert I_T -\bar{I}\rvert=3,5.10^{-3}$

Do đó $\lvert I-\bar{I}\rvert\leqslant \lvert I-I_T\rvert+\lvert I_T-\bar{I}\rvert\leqslant 0,0295$



d) $I=\int_{0}^{6} \frac{1}{x^2+1}\mathrm{d}x$

Ta có $h=\frac{6-0}{10}=\frac{3}{5}$

\textbf{Áp dụng công thức Simpson 1/3}

Ta có bảng sau:
\begin{center}
\begin{tabular}{|c|c|c|c|c|}
\hline
$i$&$x_i$&\multicolumn {3} {|c|} {$y_i=f(x_i)= \frac{1}{x^2+1}$}\\ \hline
0&0&1& & \\ \hline
1&$0,6$& &0,7353& \\ \hline
2&$1,2$& & &0,4098 \\ \hline
3&$1,8$& &0,2358& \\ \hline
4&$2,4$& & &0,1479 \\ \hline
5&$3$& &$0,1$& \\ \hline
6&$3,6$& & &0,0716 \\ \hline
7&$4,2$& &0,0536& \\ \hline
8&$4,8$& & &0,0416 \\ \hline
9&$5,4$& & 0,0332& \\ \hline
10&$6$&0,0270& & \\ \hline
\end{tabular}
\end{center}

Theo công thức Simpson 1/3, ta có:

$I_S\approx \frac{1}{5}.\left[1,0270+4.1,1579+2.0,6980\right]\approx 1,410973$


Đánh giá sai số:

$$\lvert I-I_S \rvert\leqslant \frac{\max_{0\leqslant x\leqslant 6}\left\lvert f^{(4)}(x)\right\rvert}{180}(6 -0)h^4=\frac{24}{180}.6.\left( \frac{3}{5}\right)^4= 0,10368  $$

\textbf{Áp dụng công thức hình thang}

Ta có bảng sau:
\begin{center}
\begin{tabular}{|c|c|c|c|c|}
\hline
$i$&$x_i$&\multicolumn {2} {|c|} {$y_i=f(x_i)= \frac{1}{x^2+1}$}\\ \hline
0&0&1&  \\ \hline
1&$0,6$& &$\frac{25}{34}$ \\ \hline
2&$1,2$&  &$\frac{25}{61}$ \\ \hline
3&$1,8$& &$\frac{25}{106}$ \\ \hline
4&$2,4$&  &$\frac{25}{169}$ \\ \hline
5&$3$& &$\frac{1}{10}$ \\ \hline
6&$3,6$&  &$\frac{25}{349}$ \\ \hline
7&$4,2$& &$\frac{25}{466}$ \\ \hline
8&$4,8$&  &$\frac{25}{601}$ \\ \hline
9&$5,4$& &$\frac{25}{754}$ \\ \hline
10&$6$&$\frac{1}{37}$& \\ \hline
& &$\frac{38}{37}$&$\frac{11967477}{6543383}$\\ \hline
\end{tabular}
\end{center}

Do đó giá trị gần đúng của tích phân đã cho là:
$$I_T\approx \frac{6-0}{2.10}.(\frac{38}{37}+2.\frac{11967477}{6543383})=1,40547$$

*Đánh giá sai số

Ta có $M=\max_{0\leqslant 6}\lvert f''(x)\rvert= 2$ và $\bar{I}=1,41$

nên $\lvert I_T -\bar{I}\rvert\leqslant \frac{M}{12}.(6 -0).\left(\frac{3}{5}\right)^2=0,36$

và $\lvert I_T -\bar{I}\rvert=4,53.10^{-3}$

Do đó $\lvert I-\bar{I}\rvert\leqslant \lvert I-I_T\rvert+\lvert I_T-\bar{I}\rvert\leqslant 0,364653$





f) $I= \int\limits^{4}_{2} \frac{1}{\left(x-1\right)^2}\mathrm{d}x$\\
Ta có $h=\frac{b-a}{n}=\frac{4-2}{10}=0,2$ và $f\left(x\right)=\frac{1}{\left(x-1\right)^2}$\\
\textbf{Áp dụng công thức hình thang}\\
Ta có bảng sau:\\
\begin{center}\begin{tabular}{|c|c|c|}
	\hline
	$x_i$ & \multicolumn{2}{|c|}{$y_i=f\left(x_i\right)=\frac{1}{\left(x_i-1\right)^2}$}\\
	\hline
	$2,0$ & $1$ & \\ \hline
	$2,2$ && $25/36$\\ \hline
	$2,4$ && $25/49$\\ \hline
	$2,6$ && $25/64$\\ \hline
	$2,6$ && $25/81$\\ \hline
	$3,0$ && $1/4$\\ \hline
	$3,2$ && $25/121$\\ \hline
	$3,4$ && $25/144$\\ \hline
	$3,6$ && $25/169$\\ \hline
	$3,8$ && $25/196$\\ \hline
	$4,0$ & $1/9$ & \\ \hline
	 & $10/9$ & $2,809618197$\\ \hline
\end{tabular}\end{center}\\
Vậy theo công thức hình thang ta tính được giá trị gần đúng của tích phân là:\\
$I\approx \int\limits^{4}_{2} \frac{1}{\left(x-1\right)^2}\mathrm{d}x=\frac{4-2}{2\left(10\right)}\left(\frac{10}{9}+2\left(2,809618197\right) \right) = 0,6730347505$\\
Nếu làm tròn đến năm chữ số thập phân thì $I_T = 0,67303$\\
Đánh giá sai số theo công thức tích phân, ta có:\\
$f'\left(x\right)=-\frac{2}{\left(x-1\right)^3}$; $f''\left(x\right)=\frac{6}{\left(x-1\right)^4}$\\
Do hàm $f''$ nghịch biến trên đoạn $\left[2;4\right]$ nên $ M= \underset{2\leq x \leq 4}{\max}\left|f''\left(x\right)\right| = \left|f''\left(2\right)\right|=6$ \\
Nên $\left| I - I_T\right| \leq \frac{6}{12} \left(4-2\right)\left(0,2\right)^2= 0,04$\\
Và $\left|I_T- \overline{I} \right| = 4,7505.10^{-6}$\\
Do đó\\
$\left| I - \overline{I} \right| \leq \left|I - I_T\right| + \left|I_T - \overline{I}\right| = 0,04 + 4,7505.10^{-6}$\\
\textbf{Áp dụng công thức Simpson 1/3}\\
Ta có bảng:\\
\begin{center}\begin{tabular}{|c|c|c|c|c|}
	\hline
	$i$ & $x_i$ & \multicolumn{3}{|c|}{$f\left(x_i\right)=\frac{1}{\left(x_i - 1 \right)^2}$}\\ \hline
	$0$ & $2,0$ & $1$ & &\\ \hline
	$1$ & $2,2$ & & $25/36$ &\\ \hline
	$2$ & $2,4$ & & & $25/49$ \\ \hline
	$3$ & $2,6$ & & $25/64$ & \\ \hline
	$4$ & $2,8$ & & & $25/81$ \\ \hline
	$5$ & $3,0$ & & $1/4$ & \\ \hline
	$6$ & $3,2$ & & & $25/121$ \\ \hline
	$7$ & $3,4$ & & $25/144$ &\\ \hline
	$8$ & $3,6$ & & & $25/169$ \\ \hline
	$9$ & $3,8$ & & $25/196$ & \\ \hline
	$10$ & $4,0$ & $1/9$ & & \\ \hline
	& & $10/9$ & $1,636231576$ & $1,173386621$ \\ \hline
\end{tabular}\end{center}
Áp dụng công thức Simpson 1/3 ta tính gần đúng tích phân là:\\
$I_S=\int\limits^{4}_{2} \frac{1}{\left(x-1\right)^2}\mathrm{d}x \approx \frac{0,2}{3} \left[ \frac{10}{9} + 4\left(1,6366231576\right)+ 2\left(1,173386621\right) \right]=0,6668540438$\\
Nếu lấy 5 chữ số thập phân, khi đó $\overline{I}=0,66685$. Nên $\left| I_S - \overline{I} \right| = 4.0438\times 10^{-6}$\\
Đánh giá sai số theo công thức, ta có:\\
$f^{\left(3\right)}\left(x\right) =\frac{24}{\left(x-1\right)^5}$; $f^{\left(4\right)}\left(x\right)=\frac{120}{\left(x-1\right)^6}$\\
Do $f^{\left(4\right)}\left(x\right)$ là hàm nghịch biến trên đoạn $\left[2;4\right]$ nên $M=\underset{2 \leq x \leq 4}{\max} \left| f^{\left(4\right)}\left(x\right) \right| = \left| f^{\left(4\right)}\left(2\right) \right|= 120$\\
Do đó $\left| I - I_{S} \right| \leq \frac{120}{180}\times \left(4-2\right) \times \left(0,2\right)^4=0,05\left(3\right) $\\
Vậy $\left| I-\overline{I}\right| \leq \left| I- I_S\right| + \left|I_S -\overline{I}\right|=0,05\left(3\right) - 4.0438\times 10^{-6}$\\



j) $I=\int\limits_{0,1}^{1,1} \frac{1}{\left(1+4x\right)^2}\mathrm{d}x$\\
Ta có $h=\frac{1,1-0,1}{10}=0,1$ và $g\left(x\right)=\frac{1}{\left(1+4x\right)^2}$\\
Ta tìm được các đạo hàm của $g\left(x\right)$\\
$g'\left(x\right) = -\frac{8}{\left(1+4x\right)^3}$\\
$g''\left(x\right) = \frac{96}{\left(1+4x\right)^4}$\\
$g^{\left(3\right)\left(x\right)} = -\frac{1536}{\left(1+4x\right)^5}$\\
$g^{\left(4\right)}\left(x\right) =\frac{30720}{\left(1+4x\right)^6}$\\
\textbf{Áp dụng công thức hình thang}\\
Ta có bảng giá trị:\\
\begin{center}\begin{tabular}{|c|c|c|c|}
	\hline
	& $x_i$ & \multicolumn{2}{|c|}{$y_i=g\left(x_i\right)= \frac{1}{\left(1+4x_i\right)^2}$}\\ \hline 
	$0$ & $0,1$ & $25/49$ & \\ \hline
	$1$ & $0,2$ & & $25/81$ \\ \hline
	$2$ & $0.3$ & & $25/121$\\ \hline
	$3$ & $0,4$ & & $25/169$\\ \hline
	$4$ & $0,5$ & & $1/9$\\ \hline
	$5$ & $0,6$ & & $25/289$ \\ \hline
	$6$ & $0,7$ & & $25/361$ \\ \hline
	$7$ & $0,8$ & & $25/441$\\ \hline
	$8$ & $0,9$ & & $25/529$\\ \hline
	$9$ & $1,0$ & & $1/25$\\ \hline
	$10$ & $1,1$ & $25/729$ & \\ \hline
	& & $0,5444976344$ & $1,03399924$ \\ \hline
\end{tabular}\end{center}
Vậy theo công thức hình thang, giá trị gần đúng của tích phân cần tìm là:\\
$I_T=\int\limits_{0,1}^{1,1} g\left(x\right)\mathrm{d}x \approx \frac{0,1}{2}\left[0,5444976344 + 2\left(1,03399924\right) \right] =0,1306248057$\\
Nếu làm tròn đến năm chữ số thập phân thì $\overline{I}=0,13062$.\\
Đánh giá sai số theo công thức tích phân:\\
Ta có
\begin{center}
$$M=\underset{0,1 \leq x \leq\ 1,1}\max \left|g''\left(x\right) \right|= \left|g'' \left(1,1\right) \right|= 0,1129005854$$
\end{center}
Nên $$\left|I-I_T\right|\leq \frac{M}{12}\left(1,1-0,1\right)\left(0,1\right)^2=9,408382116\times 10^{-5}$$\\ và $$\left|I_T-\overline{I}\right| =4,8075\times 10^{-6}$$\\
Do đó\\
$$\left|I-\overline{I}\right| \leq \left| I- I_T \right| + \left| I_T -\overline{I} \right| \leq 9,408382116\times 10^{-5} + 4,8075\times 10^{-6} $$